\documentclass{article}
\usepackage[utf8]{inputenc}
\usepackage{enumitem}
\usepackage{lipsum} % For generating placeholder text
\usepackage{amsmath}
\usepackage{tasks}
\usepackage{geometry}
\geometry{a4paper, margin=.5in}
\setlength{\parindent}{0pt}
\usepackage{caption}
\usepackage{graphicx}
\usepackage{xcolor}
\usepackage{hyperref}

% Customize your exam header here
% \newcommand{\examtitle}{Final Exam || PHYS2003}
% \newcommand{\examdate}{Date: Dec 19, 2023}
% \newcommand{\examduration}{Time: 120 min.}
% \newcommand{\instructorname}{Pranjal Agarwal}
% \newcommand{\collegename}{Baruch College,CUNY}
% \newcommand{\totalpoints}{10 points}
% \newcommand{\coursename}{PHYS2003, Fall 2023 }

\title{\examtitle}
\author{}
\date{}

\begin{document}

\maketitle

\begin{center}
\begin{tabular}{ll}
% \textbf{Date:}       & \examdate \\
% \textbf{Duration:}       & \examduration \\
% \textbf{Instructor:}       & \instructorname \\
% % \textbf{College:} & \collegename \\
% \textbf{Total Points:} & \totalpoints \\
% \textbf{Course:} & \coursename \\
\end{tabular}
\end{center}

\section{Background}

Symmetrically Informationally complete POVM (SIC POVM) for qutrits is used for doing tomography on the unknown paired states. The only difference is that instead of qutrits, we instead have three dimensional symmetric subspace of two qubits. The basis elements and their (ordered) matrix representation are : 

$$| 00 \rangle,| 11 \rangle,\frac{1}{\sqrt{2}}| 00 + 11\rangle \quad  \longrightarrow   \quad   \begin{pmatrix}
1 \\
0 \\
0
\end{pmatrix} \begin{pmatrix}
0 \\
1 \\
0
\end{pmatrix}, \begin{pmatrix}
0 \\
0 \\
1
\end{pmatrix} $$ 

In the above representation the qutrit SIC POVM has nine elements, represented by the column vectors of the following matrix:

\begin{equation}
\frac{1}{\sqrt{2}}\left(\begin{array}{ccccccccc}
0 & -1 & 1 & 0 & -1 & 1 & 0 & -1 & 1 \\
1 & 0 & -1 & \omega & 0 & -\omega & \omega^2 & 0 & -\omega^2 \\
-1 & 1 & 0 & -\omega^2 & \omega^2 & 0 & -\omega & \omega & 0
\end{array}\right)
\end{equation}

where, $\omega = e^{i\frac{2\pi}{3}}$. \\ 

The SIC POVM, using the above directions $| v_i \rangle $, is:

\begin{equation}
    \Pi = \{ \Pi_i \} \quad ; \quad \Pi_i = \frac{1}{3}| v_i \rangle \langle v_i|
\end{equation} 

% \begin{equation}
%     \Pi = \{ \Pi_i \}   \quad ; \qqaud  \Pi_i = \frac{1}{3} \v_i \rangle v_i | 
% \end{equation}



We want to get an extension of the above measurement such that it is potentially implementable using a suitable apparatus. We know that a physical measurement has to have mutually exclusive outcomes, thus in matrix representation the direction vectors must be mutually othogonal, along with being normalized. They must also reproduce the same measurements when measured/projected in the three dimensional subspace of the SIC POVM which is same as the symmetric subspace in our particular problem.
Thus, overall, we want 9 mutually orthonormal vectors with their subspace projections equal to the corresponding SIC POVM direction measurement .

Given the above, to figure out the normalization constant we will split the full 9d vector $| \Tilde{v_i} \rangle$ into the known POVM direction $|v_i \rangle$ and unknown part $|w_i \rangle$. 

Mathematically, the above translates to :


\[
\begin{pmatrix}
1 & 0 & 0 & 0 & 0 & 0 & 0 & 0 & 0 \\
0 & 1 & 0 & 0 & 0 & 0 & 0 & 0 & 0 \\
0 & 0 & 1 & 0 & 0 & 0 & 0 & 0 & 0 \\
0 & 0 & 0 & 0 & 0 & 0 & 0 & 0 & 0 \\
0 & 0 & 0 & 0 & 0 & 0 & 0 & 0 & 0 \\
0 & 0 & 0 & 0 & 0 & 0 & 0 & 0 & 0 \\
0 & 0 & 0 & 0 & 0 & 0 & 0 & 0 & 0 \\
0 & 0 & 0 & 0 & 0 & 0 & 0 & 0 & 0 \\
0 & 0 & 0 & 0 & 0 & 0 & 0 & 0 & 0
\end{pmatrix}  
\begin{pmatrix}
\tilde{v}_1j \\
\tilde{v}_2j \\
\tilde{v}_3j \\
\tilde{v}_4j \\
\tilde{v}_5j \\
\tilde{v}_6j \\
\tilde{v}_7j \\
\tilde{v}_8j \\
\tilde{v}_9j
\end{pmatrix}  =  \frac{1}{\sqrt{3}} \begin{pmatrix}
 \frac{1}{\sqrt{2}} v_{1j} \\
 \frac{1}{\sqrt{2}} v_{2j} \\
 \frac{1}{\sqrt{2}}v_{3j} \\
0 \\
0 \\
0 \\
0 \\
0 \\
0
\end{pmatrix} 
\implies \Tilde{v_{1j}} = \frac{1}{\sqrt{6}} v_{1j}, \Tilde{v_{2j}} = \frac{1}{\sqrt{6}} v_{2j}, \Tilde{v_{3j}} = \frac{1}{\sqrt{6}} v_{3j}
\]

Thus, a general 9D vector corresponding to POVM direction $|v_j\rangle$ is of the form:
\begin{equation}
    
|\Tilde{v}_{j}\rangle = \frac{1}{\sqrt{6}} \begin{pmatrix}
v_{1j} \\
v_{2j} \\
v_{3j} \\
c_{4j} \\
c_{5j} \\
c_{6j} \\
c_{7j} \\
c_{8j} \\
c_{9j}
\end{pmatrix}
\end{equation}

where, $c_{ij}$ are unknown, $v_{ij}$ are from the 3d POVM vector $|v_j\rangle$ and $\langle \Tilde{v}_{j}|\Tilde{v}_{j}\rangle = 1$.


Now we want to find the unknown complex numbers $c_{ij}$ such that the 9D vectors are mutually orthonormal. 
This means that the dot product of any two vectors should be zero. The total number of such dot products is 36. 
The number of the unknown entries is 6*9 = 54. But out of those, 9 are normalization constants, so we have 45 unknowns overall. 
36 equations and 45 unknowns means that we have freedom to potentially set at least 9 unknowns as desired. Out of many choices possible, one 
preferred choice is to set as many unknowns of first two vectors zero as possible. So for the first vector, we set $c_{4j}$ as the normalization 
element, and others set to zero. For the second vector, we find $c_{4j}$ using the orthogonality condition, find $c_{5j}$ using normalization condition, 
and set the rest to zero. Thus we have made choices for 5+4 = 9 unknowns, resulting in the first two vectors becoming:



% (1/sqrt6)*(0,1,-1,2,0...)
\begin{equation}
    |\Tilde{v}_{1}\rangle = \frac{1}{\sqrt{6}} \begin{pmatrix}
0 \\ 1 \\ -1 \\ 0 \\ 0 \\ 0 \\ 0 \\ 0 \\ 0
\end{pmatrix} \quad ; \quad 
|\Tilde{v}_{2}\rangle = \frac{1}{\sqrt{6}} \begin{pmatrix}
    -1 \\ 0 \\ 1 \\ \frac{1}{2} \\ \frac{\sqrt{15}}{2} \\ 0 \\ 0 \\ 0 \\ 0
\end{pmatrix}
\end{equation}


Knowing these two vectors, we can now find the $c_{4j}$ and $c_{5j}$ for the rest of the
vectors by making use of orthonormality equations with respect to the first two vectors. Doing these calcuations, the vectors become:
% create a nine by nine matrix with 1/sqrt6 on the outside. The first three rows are simply the 3d v vectors


% \[
% \begin{pmatrix}
% 0       & -1       & 1                  & 0                          & -1                             & 1                          & 0                           & -1                       & 1 \\
% 1       & 0        & -1                 & \omega                     & 0                              & -\omega                    & \omega^2                    & 0                        & -\omega^2 \\
% -1      & 1        & 0                  & -\omega^2                  & \omega^2                       & 0                          & -\omega                     & \omega                   & 0 \\
% 2       & .5       & .5                 & .5                         & \omega^2/2                     & \omega/2                   & -(\omega+\omega^2)/2        & \omega/2                 & \omega^2/2 \\
% 0  & \sqrt(15)/2   & 3/(2*\sqrt(15))    & (2\omega^2-.5)/\sqrt(15)   & -(2+5\omega^2/2)/\sqrt(15)     & (2-\omega/2)/sqrt(15)      & (2\omega-.5)/\sqrt(15)      & -(2+5\omega/2)/\sqrt(15) & (2-\omega^2/2)/\sqrt(15) \\
% 0       & 0        & c_{63}             & c_{64}                     & c_{65}                         & c_{66}                     & c_{67}                      & c_{68}                   & c_{69} \\
% 0       & 0        & c_{73}             & c_{74}                     & c_{75}                         & c_{76}                     & c_{77}                      & c_{78}                   & c_{79} \\
% 0       & 0        & c_{83}             & c_{84}                     & c_{85}                         & c_{86}                     & c_{87}                      & c_{88}                   & c_{89} \\
% 0       & 0        & c_{93}             & c_{94}                     & c_{95}                         & c_{96}                     & c_{97}                      & c_{98}                   & c_{99} \\
% \end{pmatrix}
% \]


% chatgpt corrected one
\[
\begin{pmatrix}
0 & -1 & 1 & 0 & -1 & 1 & 0 & -1 & 1 \\
1 & 0 & -1 & \omega & 0 & -\omega & \omega^2 & 0 & -\omega^2 \\
-1 & 1 & 0 & -\omega^2 & \omega^2 & 0 & -\omega & \omega & 0 \\
2 & 0.5 & 0.5 & 0.5 & \frac{\omega^2}{2} & \frac{\omega}{2} & -\frac{\omega + \omega^2}{2} & \frac{\omega}{2} & \frac{\omega^2}{2} \\
0 & \frac{\sqrt{15}}{2} & \frac{3}{2\sqrt{15}} & \frac{2\omega^2 - 0.5}{\sqrt{15}} & -\frac{2 + 5(\omega^2 / 2)}{\sqrt{15}} & \frac{2 - \omega / 2}{\sqrt{15}} & \frac{2\omega - 0.5}{\sqrt{15}} & -\frac{2 + 5(\omega / 2)}{\sqrt{15}} & \frac{2 - \omega^2 / 2}{\sqrt{15}} \\
0 & 0 & c_{63} & c_{64} & c_{65} & c_{66} & c_{67} & c_{68} & c_{69} \\
0 & 0 & c_{73} & c_{74} & c_{75} & c_{76} & c_{77} & c_{78} & c_{79} \\
0 & 0 & c_{83} & c_{84} & c_{85} & c_{86} & c_{87} & c_{88} & c_{89} \\
0 & 0 & c_{93} & c_{94} & c_{95} & c_{96} & c_{97} & c_{98} & c_{99} \\
\end{pmatrix}
\]




Thus, at this point we have 6+5+4+3+2+1 = 21 orthogonality and 7 normalization equations, and 7*4 = 28 unknowns.

\textcolor{red}{Since there are non-linear equations present, more than one solutions can exist for this system.} We can find one such solution
by numerically solving the equations. We can computationally find such solutions to a very high degree of accuracy. To do that, we define the Gram 
matrix as the matrix of dot products of the 9D vectors:
\begin{equation}
    G_{ij} = \langle v_i, v_j \rangle
\end{equation}

Here the diagonal elements are supposed to be $ \langle v_i, v_i \rangle = 1$ (normalization) and the off-diagonal elements are supposed to be
$ \langle v_i, v_j \rangle = 0$ (orthogonality). One obvious quantity that we can minimize in order to find the solutions 
is $\sum_{i,j} (G_{ij} - I_{ij})$. This ensures that at the minimum values all the diagonal elements of the Gram matrix are 1 and all 
the off-diagonal elements are 0, as required. 


% Turns out that directly brute-force solving these equations is not computationally feasible. 
% The best solution found in that way, using  turned out to roughly give average angle between the vectors as 87.6 degrees, 
% which is not very close to the expected 90 degrees. With more compute power, one could potentially find a better solution there but it was 
% thought better to find a more suitable method. 

% Thats where the idea of using Lagrange multipliers comes in. So using that we want to basically reduce the problem to 
% a system of simpler equations which are easier to solve computationally. 


% \section{Simplifying using Lagrange multipliers method}


% We use Lagrange multipliers to simplify the equations and find the unknown complex numbers \(c_{ij}\). 
% First, to set the objective function/Lagrange function, we notice that we want something which wehn minimized corresponds to the 
% orthogonality and normalization conditions. additionally, we want to minimize the computational complexity of the resulting partial derivative equations, 
% which is the reason we want to use this method. 
% One such Lagrangian could be the sum of the residues of the orthogonality and normalization equations.
% In that case, the one of the extrema of the Lagrangian may correspond to the zero of the Lagrangian, which would imply that the orthogonality and
% normalization conditions are satisfied. This can be checked once the solution is found by substituting back in the Lagrangian and into the equations.
% Mathematically:

% The residues  $r_{ij}$, $n_{i}$  represents the difference between the calculated inner product and the expected value (which is 0 for orthogonal vectors and 1 for the same vector).

% $$r_{ij} = \langle \tilde{v}_i | \tilde{v}j \rangle - \delta{ij}$$
% $$n_i = \langle \tilde{v}_i | \tilde{v}_i \rangle - 1$$

% where  \delta_{ij}  is the Kronecker delta.

% We can split the 9d vectors into the known and unknown parts as below. the lagraingian as sum of the above residues then becomes:

% \begin{equation}
%     \tilde{v}_i = 
% \[
% \mathcal{L} = \sum_{i=1}^{9} \sum_{j=1}^{9} \left( \langle \tilde{v}_i | \tilde{v}_j \rangle - \delta_{ij} \right)^2 + \sum_{i=1}^{9} \lambda_i \left( \langle \tilde{v}_i | \tilde{v}_i \rangle - 1 \right)
% \]

% where \(\lambda_i\) are the Lagrange multipliers. We take the partial derivatives of \(\mathcal{L}\) with respect to \(c_{ij}\) and \(\lambda_i\) and set them to zero to find the critical points.

% After solving the system of equations, we obtain the following simplified equations:

% \[
% \begin{aligned}
% \frac{\partial \mathcal{L}}{\partial c_{ij}} &= 2 \sum_{k=1}^{9} \left( \langle \tilde{v}_i | \tilde{v}_k \rangle - \delta_{ik} \right) \langle \tilde{v}_k | c_{ij} \rangle + 2 \lambda_i \langle \tilde{v}_i | c_{ij} \rangle = 0, \\
% \frac{\partial \mathcal{L}}{\partial \lambda_i} &= \langle \tilde{v}_i | \tilde{v}_i \rangle - 1 = 0.
% \end{aligned}
% \]

% This system of equations is solved iteratively to find the values of \(c_{ij}\) that satisfy both the orthogonality and normalization constraints.


\section{Optical Implementation}

\end{document}


